%%%%%%%%%%%%%%%%%%%%%%%%%%%%%%%%%%%%%%%%%%%%%%%%%%%%%%%%%%%%
% standard header
    \documentclass[12pt,fleqn]{article}
    \usepackage[top=2cm,bottom=2cm,left=2cm,right=2cm]{geometry}
    \usepackage[utf8x]{inputenc}
    \usepackage{titlesec} % see below
    \usepackage{xcolor} % \color \textcolor
    \usepackage{hyperref} % \href
    \usepackage[normalem]{ulem} % normalem retains \emph as italic
    % \uline \uuline \uwave \sout
    \usepackage{enumitem}
    % \begin{enumerate}[label=\Alpha*.]
    % \begin{enumerate}[label=\(\bullet\)]
    \usepackage{graphicx}
    % \includegraphicswidth=0.5\textwidth,trim=[0cm 0cm 0cm 0cm,clip]{file.png}
    \usepackage{pdfpages} % \includepdf[pages=1]{file.pdf}
    % for prose
    %\usepackage[doublespacing]{setspace}
    \usepackage{csquotes} % \blockquote
    %%%%%%%%%%%%%%%%%%%%%%%%%%%%%%%%%%%%%%%%%%%%%%%%%%%%%%%%%%%%
    \renewcommand{\rmdefault}{cmr}
    \renewcommand{\sfdefault}{cmss}
    \renewcommand{\ttdefault}{cmtt}
    \renewcommand{\familydefault}{\rmdefault}
    \setlength{\titlewidth}{\textwidth}

    \titlespacing{\section}{0pt}{0pt}{0pt}
    \titlespacing{\subsection}{0pt}{0pt}{0pt}
    \titlespacing{\subsubsection}{0pt}{0pt}{0pt}
    \titlespacing{\paragraph}{0pt}{0pt}{0pt}
    % https://www.overleaf.com/learn/latex/How_to_write_a_LaTeX_class_file_and_design_your_own_CV_(Part_1)
    % \titleformat{command}[shape]{format}{label}{sep}{before-code}[after-code]
    \titleformat{\section}         % Customise the \section command 
        [hang]
        {\Large\ttfamily\raggedright} % Make the \section headers large (\Large),
                                   % small capitals (\scshape) and left aligned (\raggedright)
        {}{0em}                      % Can be used to give a prefix to all sections, like 'Section ...'
        {}                           % Can be used to insert code before the heading
        [\titlerule]                 % Inserts a horizontal line after the heading
    \titleformat{\subsection}
        [hang]
        {\large\ttfamily\raggedright}
        {}{0em}
        {}
        []
%%%%%%%%%%%%%%%%%%%%%%%%%%%%%%%%%%%%%%%%%%%%%%%%%%%%%%%%%%%%
% packages
    \usepackage{hejohns-hazel}
%%%%%%%%%%%%%%%%%%%%%%%%%%%%%%%%%%%%%%%%%%%%%%%%%%%%%%%%%%%%
% get rid of 10-modules stuff
    \renewcommand*{\TypVarCtx}{\Phi}
    \renewcommand*{\ModVarCtx}{}
    \renewcommand*{\SigVarCtx}[1][]{}
% misc
    \newcommand*{\AlgConsistentSubKindOp}{%
        \begin{tikzpicture}[scale=0.5, baseline=-1.5mm, thick]
        \node(triangle) at (-0.3mm, 0){\Large$\triangleleft$};
        \node(sim) at (0, -4.5mm){$\sim$};
        \end{tikzpicture}%
    }
    \newcommand*{\AlgTypeEquivOp}{%
        \begin{tikzpicture}[scale=0.5, baseline=-1.2mm, thick]
        \node(equiv) at (0, 0){$\equiv$};
        \node(lb) at (-5mm, .1mm){$<$};
        \node(rb) at (5mm, .1mm){$>$};
        \end{tikzpicture}%
    }
\usepackage{listings}
\pagenumbering{gobble}
\nonfrenchspacing
%\frenchspacing % when monospaced
\begin{document}
\title{Hazel MetaPhi: 9-type-aliases}
\author{}
\date{\today}
\maketitle
\section{How to read}
    \subsection*{}
    \begin{tabular}{rl}
        \red{800000} & \red{kinds} \\
        \green{008000} & \green{types (constructors)} \\
        \blue{000080} & \blue{terms} \\
    \end{tabular}
\section{Issues}
    \subsection*{}
    \begin{enumerate}[label=Issue \arabic*:]
        \item Algorithmic rules are not ``officially" algorithmic \\
            Since we are going all out to Do It Right™, it should be noted that the declarative/algorithmic bifurcation is not complete with $\KndAsc$ ($\WellFormedAtKind{\typ}$). \\
            For example, kind analysis is premissed on $\ConsistentSubKindOp$ ($\ConsistentSubKind{\knd[_1]}{\knd[_2]}$) at \texttt{KAASubsume},
            which is itself premissed on $\TypeEquivOp$ ($\TypeEquiv{\typ[_1]}{\typ[_2]}[]$) at \texttt{KCRespectEquiv},
            which is itself premissed on $\KndAsc$ ($\WellFormedAtKind{\typ}$) at \texttt{KCESingEquiv}. \\
            Explicitly algorithmic counterparts to $\ConsistentSubKindOp$ and $\TypeEquivOp$ need to be defined.
            Suggested notation: $\AlgConsistentSubKindOp$ and $\AlgTypeEquivOp$ (faintly reminiscent of Stone and Harper).
            A declarative most specific kind (up to equivalence) would also be useful; \emph{principal kinds} ($\Uparrow$) a la Stone and Harper.
        \item A $\typvar\KndAssump\knd\in\TypVarCtx \implies \typvar\KndAsc\SKind{\typvar}$ (like) rule is problematic \\
            When $\forall \typ.\WellFormedAtKind{\typ}\implies \WellFormedAtKind{\typ}[\Type]$, as is now,
            this is fine (except for improperly (non)contexted holes $\forall \typ.\WellFormedAtKind[]{\typ}[\Type]$ I would think).
            But in the presence of non$\Type$ kinds, this rule no longer conveys the correct meaning (intuitively speaking). \\
            For example, if we redefined $\greentt{list(\typ)}$ to be a type constructor proper instead of a builtin type schema,
            such that $\WellFormedAtKind[\greentt{List\KndAsc\Type\red{\rightarrow}\Type}\in\TypVarCtx]{\greentt{List}}[\Type \red{\rightarrow} \Type]$,
            with the following definitions, \\
            $\mathtt{data~List~\alpha = Nil~|~Cons~\alpha~(List~\alpha)}$ \\
            $\mathtt{type~T = List}$ \\
            We would expect the following to be derivable: \\
            $\TypeEquiv[\cdot;\green{List}\KndAssump\red{\Type\rightarrow\Type}, \green{T}\KndAssump\SKind[\red{\Type\rightarrow\Type}]{\green{List}} := \red{\DepFunKind{\typvar\KndAnn\Type}{\SKind{\green{List~\typvar}}}}]{\greentt{T}}{\greentt{List}}[]$ \\
            as well as: \\
            $\TypeEquiv[\cdot;\green{List}\KndAssump\red{\Type\rightarrow\Type}, \green{T}\KndAssump\SKind[\red{\Type\rightarrow\Type}]{\green{List}} := \red{\DepFunKind{\typvar\KndAnn\Type}{\SKind{\green{List~\typvar}}}}, \typvar[_{\alpha}]\KndAssump\Type]{\greentt{T~\typvar[_{\alpha}]}}{\greentt{List~\typvar[_{\alpha}]}}[]$ \\
            but we would not want: \\
            $\WellFormedAtKind[\cdot;\green{List}\KndAssump\red{\Type\rightarrow\Type}, \green{T}\KndAssump\SKind[\red{\Type\rightarrow\Type}]{\green{List}} := \red{\DepFunKind{\typvar\KndAnn\Type}{\SKind{\green{List~\typvar}}}}]{\greentt{T}}[\SKind{\greentt{T}}]$ \\
            However as the notation suggests: \\
            $\WellFormedAtKind[\cdot;\green{List}\KndAssump\red{\Type\rightarrow\Type}, \green{T}\KndAssump\SKind[\red{\Type\rightarrow\Type}]{\green{List}} := \red{\DepFunKind{\typvar\KndAnn\Type}{\SKind{\green{List~\typvar}}}}]{\greentt{T}}[\SKind[{\SKind[\red{\Type\rightarrow\Type}]{\green{List}}}]{\greentt{T}}]$ \\
            is expected to be derivable. \\
            It is not possible to model this behavior without higher order singletons and dependent function kinds.
            Thankfully, Stone and Harper showed that HOSingletons are definable in terms of ``vanilla" singletons.
            Thus, we do not need to add HOSingletons to the object language itself-- only to the metalangauge.
            HOSingletons should make the metatheory proofs more general, even without considering type constructors.
            See Attachment 2 for an example.
            The question is how HOSingletons and holes should interact.
            And with type constructors, how type holes should behave since we now have nontrivial normalization semantics.
    \end{enumerate}
\section{Not Issues-- but worth mentioning}
    \subsection*{}
    \begin{enumerate}[label=Nibwm \arabic*:]
        \item In the presence of higher order types/type constructors, there exists a more nuanced notion of type equivalence (extensionality) in which equivalence depends on the kind at which the types are compared \\
            Without ``real" subkinding, examples are a bit more contrived. \\
            From Stone and Harper (2006): \\
            \[
                \TypeEquiv[]{\TypCFun{\typvar\KndAsc\Type}{\typvar}}{\TypCFun{\typvar\KndAsc\Type}{\Int}}[\SKind{\Int} \red{\rightarrow} \Type]
            \]
            should be derivable but
            \[
                \TypeEquiv[]{\TypCFun{\typvar\KndAsc\Type}{\typvar}}{\TypCFun{\typvar\KndAsc\Type}{\Int}}[\Type \red{\rightarrow} \Type]
            \]
            should not, where $\ConsistentSubKind[]{\red{\Type \rightarrow \Type}}{\red{\SKind{\Int} \rightarrow \Type}}$ \\
            Interestingly
            \[
                \TypeEquiv[]{\TypCFun{\typvar\KndAsc\Type}{\typvar}}{\TypCFun{\typvar\KndAsc\Type}{\Int}}[\SKind{\Int} \red{\rightarrow} \SKind{\Int}]
            \]
            should also be derivable, where $\ConsistentSubKind[]{\SKind{\Int} \red{\rightarrow} \SKind{\Int}}{\SKind{\Int} \red{\rightarrow} \Type}$ \\
            With ``real" subkinding, this behavior is more serious. \\
            From Aspinall (1995), using singleton types and $\greentt{Nat} \le \greentt{Int}$ (``real" subtyping):
            \[
                \TypeEquiv[]{\bluett{\lambda x\TypAsc Int.if~x \ge 0~then~x~else~2*x}}{\bluett{\lambda x\TypAsc Int.x}}[\greentt{Nat \rightarrow Int}]
            \]
            should be derivable but
            \[
                \TypeEquiv[]{\bluett{\lambda x\TypAsc Int.if~x \ge 0~then~x~else~2*x}}{\bluett{\lambda x\TypAsc Int.x}}[\greentt{Int \rightarrow Int}]
            \]
            should not. \\
            I do not believe this more nuanced view of equality buys anything for us.
    \end{enumerate}
\newpage
    \includepdf[scale=0.9,pages=1,pagecommand={\section{Attachments}\subsection*{9-type-aliases marked up with preliminary declarative statics notes}Attachment 1}, offset=0 -2cm]{kind_judgements_statics.pdf}
    \includepdf[scale=0.9,pages=2-,pagecommand={Attachment 1}]{kind_judgements_statics.pdf}
    \subsection*{Kind Analysis Soundness in 9-type-aliases}
    Attachment 2 \\
    If $\AnaKind{\typ}$, then $\WellFormedAtKind{\typ}$ \\
    Without HOSingletons: \\
    \begin{lstlisting}[mathescape=true]
    base case $\typ = \typvar$:
        (easy) case KAVar:
            $\typvar\KndAssump\knd[_1]\in\TypVarCtx$ and $\ConsistentSubKind{\knd[_1]}{\knd}$
            Thus by 43.2b, $\WellFormedAtKind{\typvar}[\knd[_1]]$
            By 43.2d, $\WellFormedAtKind{\typvar}[\knd]$
            Thus $\WellFormedAtKind{\typ}$
        (hard) case KAASubsume:
            $\SynKind{\typ}[\knd[_1]]$ and $\ConsistentSubKind{\knd[_1]}{\knd}$
            By KSVar, $\typvar\KndAssump\knd[_2]\in\TypVarCtx$ and $\knd[_1] = \SKind{\typvar}$
            Thus, $\ConsistentSubKind{\SKind{\typvar}}{\knd}$
            wts $\WellFormedAtKind{\typvar}[\SKind{\typvar}]$, requires $\WellFormedAtKind{\typvar}[\Type]$
            can show $\WellFormedAtKind{\typvar}[\Type]$ by exhaustive syntactic case + subsumption
            ($\forall\knd[_2].\ConsistentSubKind{\knd[_2]}{\Type}$)
    \end{lstlisting}
    With HOSingletons: \\
    \begin{lstlisting}[mathescape=true]
    base case $\typ = \typvar$:
        (easy) case KAVar:
            same
        (hard) case KAASubsume:
            $\SynKind{\typ}[\knd[_1]]$ and $\ConsistentSubKind{\knd[_1]}{\knd}$
            By KSVar, $\typvar\KndAssump\knd[_2]\in\TypVarCtx$ and $\knd[_1] = \SKind[{\knd[_2]}]{\typvar}$
            Thus, $\ConsistentSubKind{\SKind[{\knd[_2]}]{\typvar}}{\knd}$
            And, $\WellFormedAtKind{\typvar}[\knd[_2]]$
            Thus $\WellFormedAtKind{\typvar}[\SKind[{\knd[_2]}]{\typvar}]$
            Thus, $\WellFormedAtKind{\typvar}$
            Thus, $\WellFormedAtKind{\typ}$
    \end{lstlisting}
\end{document}
